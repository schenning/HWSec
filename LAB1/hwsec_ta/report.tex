\documentclass{article}
\usepackage[utf8]{inputenc}
\usepackage{amsmath}
\title{Report Lab 1: Hardware Security}
\author{Henning Schei}
\date{May 2016}
\usepackage{natbib}
\usepackage{graphicx}
\usepackage{listings}
\usepackage{color}

\definecolor{dkgreen}{rgb}{0,0.6,0}
\definecolor{gray}{rgb}{0.5,0.5,0.5}
\definecolor{mauve}{rgb}{0.58,0,0.82}

\lstset{frame=tb,
  language=python,
  aboveskip=3mm,
  belowskip=3mm,
  showstringspaces=false,
  columns=flexible,
  basicstyle={\small\ttfamily},
  numbers=none,
  numberstyle=\tiny\color{gray},
  keywordstyle=\color{blue},
  commentstyle=\color{dkgreen},
  stringstyle=\color{mauve},
  breaklines=true,
  breakatwhitespace=true,
  tabsize=3
}
\begin{document}

\maketitle

\begin{Introductoin}
The implementation of the permutation code p_permutation is a good target for a timing attack, because there is a dependence between the hamming weight of the value to be permuted and the time used by the code. !!bedre setning!! The is because of a aditional 32 iteration loop all the one's go through. \\ \\ From an attackers point of view, this can be used to build up a timing model where the correlation between hamming weight and time consumed are beeing exploited. Because       





\end{document}

